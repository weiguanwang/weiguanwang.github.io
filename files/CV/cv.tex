\documentclass[10pt,a4paper,roman,english]{moderncv}        % possible options include font size ('10pt', '11pt' and '12pt'), paper size ('a4paper', 'letterpaper', 'a5paper', 'legalpaper', 'executivepaper' and 'landscape') and font family ('sans' and 'roman')
\moderncvstyle{classic}                             % style options are 'casual' (default), 'classic', 'oldstyle' and 'banking'
\moderncvcolor{blue}                               % color options 'blue' (default), 'orange', 'green', 'red', 'purple', 'grey' and 'black'
%\nopagenumbers{}                                  % uncomment to suppress automatic page numbering for CVs longer than one page
\usepackage[utf8]{inputenc}                       % if you are not using xelatex ou lualatex, replace by the encoding you are using
\usepackage[scale=0.75,a4paper]{geometry}
\usepackage{babel}
\usepackage{xeCJK} % for Chinese characters %%   用XeLatex运行

% Set up footer and Page counter
\usepackage{lastpage}
\pagestyle{fancy}
\fancyhf{}
\lfoot{\small{Weiguan Wang}}
\cfoot{\small Updated on: \today}
\rfoot{\small{Page \thepage\ of \pageref*{LastPage}}}


% Define subsubsection
\renewcommand*{\sectionfont}{\Large\mdseries\upshape}
\renewcommand*{\subsectionfont}{\large\mdseries\upshape}
\newcommand*{\subsubsectionfont}{\mdseries}% New subsubsection font
\newcommand*{\subsubsectionstyle}[1]{{\subsubsectionfont\textcolor{black}{#1}}}
\makeatletter
\NewDocumentCommand{\subsubsection}{sm}{%
	\par\addvspace{1ex}%
	\phantomsection{}% reset the anchor for hyperrefs
	\addcontentsline{toc}{subsubsection}{#2}%
	\begin{tabular}{@{}p{\hintscolumnwidth}@{\hspace{\separatorcolumnwidth}}p{\maincolumnwidth}@{}}%
		\raggedleft\hintstyle{} &{\strut\subsubsectionstyle{#2}}%
	\end{tabular}%
	\par\nobreak\addvspace{0.5ex}\@afterheading}% to avoid a pagebreak after the heading
\makeatother

%----------------------------------------------------------------------------------
%            personal data
%----------------------------------------------------------------------------------
\firstname{\LARGE Weiguan}
\familyname{\LARGE WANG (王伟冠)}

%\title{}                               % optional, remove/comment the line if not wanted
\address{Department of Finance}{Shanghai University}{China}       % optional, remove/comment the line if not wanted; the "country" arguments can be omitted or provided empty
%\mobile{mobile number}                          % optional, remove/comment the line if not wanted
%\phone{phone number}                           % optional, remove/comment the line if not wanted
%\fax{fax number}                             % optional, remove/comment the line if not wanted
\email{weiguanwang@outlook.com}                               % optional, remove/comment the line if not wanted
\homepage{weiguanwang.github.io/}                         % optional, remove/comment the line if not wanted
\extrainfo{Citizenship: Chinese}                 % optional, remove/comment the line if not wanted
% \photo[64pt][0.4pt]{picture}                       % optional, uncomment the line if wanted; '64pt' is the height the picture must be resized to, 0.4pt is the thickness of the frame around it (put it to 0pt for no frame) and 'picture' is the name of the picture file
%\quote{}                                 % optional, remove/comment the line if not wanted
%

\begin{document}
%-----       resume       ---------------------------------------------------------
\makecvtitle

\section{Academic Positions}
\cventry{2021--present}{Assistant Professor of Finance}{Shanghai University}{China}{}{}


\section{Research Interests}
\cvitem{}{Financial Engineering, FinTech, Hedging, Machine Learning, Portfolio Management}

\section{Education}
\cventry{2016--2021}{Ph.D. Mathematics}{London School of Economics and Political Science}{}{}{Supervisor: Johannes Ruf \\ Thesis: Statistical Hedging with Neural Networks
\\Defence committee: Johannes Muhle-Karbe and Mihail Zervos}  % arguments 3 to 6 can be left empty
\cventry{2014--2015}{MSc. Financial Mathematics}{University College London}{}{Distinction}{Thesis: Optimal Execution Under Nonlinear Transient Market Impact Model}
\cventry{2009--2013}{BEng Automation}{Donghua University}{Shanghai}{First Class}{}


\section{Publications}

\subsection{Published and forthcoming papers in peer-reviewed journals}

\cvlistitem{Neural networks for option pricing and hedging: A literature review, \textbf{Journal of Computational Finance}, 2020. (with Johannes Ruf). [\hyperref{https://www.risk.net/journal-of-computational-finance/7659611/neural-networks-for-option-pricing-and-hedging-a-literature-review}{}{}{Journal}, \hyperref{https://papers.ssrn.com/sol3/papers.cfm?abstract_id=3486363}{}{}{SSRN}]
\\
\textit{Abstract:} Neural networks have been used as a nonparametric method for option pricing and hedging since the early 1990s. Far over a hundred papers have been published on this topic. This note intends to provide a comprehensive review. Papers are compared in terms of input features, output variables, benchmark models, performance measures, data partition methods, and underlying assets. Furthermore, related work and regularisation techniques are discussed.}

\cvlistitem{Hedging with linear regressions and neural networks. (with Johannes Ruf). \textbf{Journal of Business \& Economic Statistics}, 2022. 
	[\hyperref{https://www.tandfonline.com/doi/full/10.1080/07350015.2021.1931241}{}{}{Journal}, 
	\hyperref{https://papers.ssrn.com/sol3/papers.cfm?abstract_id=3580132}{}{}{SSRN}, \hyperref{https://github.com/weiguanwang/Hedging_Neural_Networks}{}{}{Code}]
	\\
	\textit{Abstract}: We study neural networks as nonparametric estimation tools for the hedging of options. To this end, we design a network, named HedgeNet, that directly outputs a hedging strategy. This network is trained to minimise the hedging error instead of the pricing error. Applied to end-of-day and tick prices of S\&P 500 and Euro Stoxx 50 options, the network is able to reduce the mean squared hedging error of the Black-Scholes benchmark significantly. We illustrate, however, that a similar benefit arises by simple linear regressions that incorporate the leverage effect. }


\cvlistitem{A note on spurious model selection. (with Johannes Ruf.) \textbf{Quantitative Finance}, 2022. [
	\hyperref{https://doi.org/10.1080/14697688.2022.2097120}{}{}{Journal}, 
	\hyperref{https://papers.ssrn.com/sol3/papers.cfm?abstract_id=3836631}{}{}{SSRN}, \hyperref{https://github.com/weiguanwang/Information_Leakage_in_Backtesting}{}{}{Code}]
	\\
	\textit{Abstract:} Testing the performance of statistical models with historical time series requires a careful handling of the data. 
	Even if a dataset is seemingly completely separated in an in-sample and an out-of-sample set information may be leaked. Such leakage can lead to a significant overestimation of the out-of-sample performance of a predictive model. We provide experimental evidence to illustrate how randomised data splits lead to overfitting in the presence of time series structure. The experiment is set up in the framework of option replication, with real-world and simulated data.}

\subsection{Papers submitted to peer-reviewed journals}

\cvlistitem{Risk premium principal components for the Chinese stock market. (with Jie Mao, Jingjing Shao.) 
}

\cvlistitem{基于线性回归和神经网络的期权对冲方法:以上证50ETF期权为例. (with 刘鑫)}


\subsection{Working papers}



\subsection{Work in progress}
\cvlistitem{Statistical hedging in multi-period with neural networks.}



\section{Grants}

\cvlistitem{Nation Natural Science Foundation of China for Young Researchers,  Grant no. 72201158, RMB 300,000, PI
}
\cvlistitem{Starting grant for young scholar at Shanghai University,RMB 150,000, PI}
\cvlistitem{Leading scholars scheme at Shanghai, RMB 150,000, PI}


\section{Awards and Prizes}

\cvitemwithcomment{2023}{The 16th Philosophy and Social Science Outstanding Accomplishment Award}{Shanghai}

\cvitemwithcomment{2019}{Final year Ph.D. Scholarship}{LSE}
\cvitemwithcomment{2013}{Excellent Graduate}{Donghua University}
\cvitemwithcomment{2011 \& 2012}{Academic Excellence Prize}{Donghua University}
\cvitemwithcomment{2010}{Shanghai Scholarship}{Shanghai Municipal Education Commission }
\cvitemwithcomment{2010}{University Scholarship}{Donghua University}




\section{Conferences}

\subsection{Contributed talks}

\cvlistitem{Information Leakage in Backtesting, 7th International Young Finance Scholar's Conference, in virtual, 2021 }
\cvlistitem{Hedging with Linear Regressions and Neural Networks, LSE Financial Mathematics Reading Group, 2018 \& 2019}


\subsection{Participated conferences}
\cvlistitem{12th European Summer School in Mathematical Finance, Padova, 2019 }
\cvlistitem{LSE Ph.D. Day,  London, 2018,  2019}
\cvlistitem{17th Winter School in Mathematical Finance, Lunteren, 2017}
\cvlistitem{LSE Risk and Stochastic Conference, London, 2016 \& 2017}


\section{Teaching}

%\subsection{Instructor}

\subsection{Teaching Assistant}
\cvitemwithcomment{2018--2019}{Computational Methods in Financial Maths}{LSE, Summer School}
\cvitemwithcomment{2017--2019}{Mathematical Methods}{LSE, Undergraduate}
\cvitemwithcomment{2017--2019}{Programming in C++}{LSE, MSc. Fin. Maths}


%\section{Thesis (Co-)Supervision}
%
%\subsection{Master}
%\cvitem{Name}{item description}
%
%\subsection{Bachelor}
%
%\cvitem{Name}{item description}

\section{Referee Activities}
\cvitem{}{Journal of Finance and Data Science, Journal of Commodity Markets}








\section{Industrial Experiences}

\cventry{20.12--21.01}{Quant Analyst (intern)}{Huatai Securities (华泰证券)}{Shanghai}{Fixed Income}{Constructed zero curves, implemented Z-Spread calculation, and conducted research in \\  understanding the movement of Z-Spread in Chinese fixed income market.}
\cventry{20.08--20.09}{Quant Analyst (intern)}{Qianxiang Asset Management (千象资产)}{Shanghai}{Commodity Trading}{Implemented optimal liquidation algorithms under transient market impact models.}
\cventry{20.06--20.08}{Quant Analyst (intern)}{Zheshang Securities (浙商证券)}{Shanghai}{Financial Derivatives}{Validated pricing models for exotic options including shark fin, snowball, and others.}

\section{Computer Skills}
\cvitem{}{C++, \LaTeX, Linux, Matlab, Microsoft Office, Python, R}




%\section{Computer skills}
%\cvdoubleitem{Category 1}{Comment}{Category 4}{Comment}
%
%
%\section{References}
%\begin{cvcolumns}
%  \cvcolumn{Category 1}{Comment}
%  \cvcolumn{Category 2}{Comment}
%  \cvcolumn{Category 3}{Comment}
%\end{cvcolumns}


\end{document}
